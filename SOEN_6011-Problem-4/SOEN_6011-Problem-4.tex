\documentclass[a4paper, 11pt]{report}
\usepackage[utf8]{inputenc}
\usepackage{titlesec}
\usepackage{fullpage} % changes the margin
\usepackage{graphicx} %package to manage images
\graphicspath{ {./images/} }

\begin{document}
\begin{titlepage}
\vspace*{0.7in}
\begin{center}
\begin{figure}[htb]
\begin{center}
\includegraphics[width=8cm]{univ_logo}
\end{center}
\end{figure}
\vspace*{0.3in}
\begin{Large}
\textbf{SOEN 6011 : SOFTWARE ENGINEERING PROCESSES} \\
\end{Large}
\vspace*{0.1in}
\begin{Large}
\textbf{SUMMER 2021} \\
\end{Large}
\vspace*{0.9in}
\begin{Large}
\textbf{SUPER CALCULATOR} \\
\end{Large}
\vspace*{0.9in}
\begin{Large} 


\textbf{PROBLEM - 4} \\
Error Handling, Debugger and Quality Attributes\\
\end{Large}
\vspace*{0.9in}
\rule{80mm}{0.1mm}\\
\vspace*{0.1in}
\begin{large}
Authors \\
\vspace*{0.1in}
Rokeya Begum Keya\\
\vspace*{0.1in}
Kyle Taylor Lange\\
\vspace*{0.1in}
Sijie Min\\
\vspace*{0.1in}
Manimaran Palani\\ 
\vspace*{0.3in}
\date{\normalsize\today} 
\end{large}
\end{center}
\begin{center}
https://www.overleaf.com/project/610304de4e6b8d24f7c781b6\end{center}
\end{titlepage}
\pagebreak
\section*{\centering{PROBLEM 4 - F2: $tan(x)$}}
\normalsize {SOEN 6011 - Summer 2021} \hfill \textbf{Rokeya Begum Keya} \\
\textbf{ Software Engineering Processes}  \hfill \textbf{40183615} \\
\hfill Repository address : https://github.com/Dakatsu/SOEN6011Calculator
\\\\\\

\pagebreak

\section*{\centering{PROBLEM 4 - F3: Hyperbolic Sine, $sinh(x)$}}
\normalsize {SOEN 6011 - Summer 2021} \hfill \textbf{Kyle Taylor Lange} \\
\textbf{ Software Engineering Processes}  \hfill \textbf{27627696} \\https://www.overleaf.com/project/610304de4e6b8d24f7c781b6
\hfill Repository address : https://github.com/Dakatsu/SOEN6011Calculator

\pagebreak

\section*{\centering{PROBLEM 4 - F5}}
\normalsize {SOEN 6011 - Summer 2021} \hfill \textbf{Sijie Min} \\
\textbf{ Software Engineering Processes}  \hfill \textbf{401*****} \\
\hfill Repository address : https://github.com/Dakatsu/SOEN6011Calculator
\\\\\\\\\\
 \begin{center} Team please add your content here \end{center}
\pagebreak

\section*{\centering{PROBLEM 4 - F7 : \(x^y\)}}
\normalsize {SOEN 6011 - Summer 2021} \hfill \textbf{Manimaran Palani} \\
\textbf{ Software Engineering Processes}  \hfill \textbf{40167543} \\
\hfill Repository address : https://github.com/Dakatsu/SOEN6011Calculator
\\
\section*{\textbf{Problem 4 - Description}}
This section presents an overview of the source code of the Super Calculator application and the
practices followed during the development.
\section*{Error Handling}
\section*{Debugger}
Eclipse has a standard debugger which allows the program to open in debug mode.It supports both step by step debugging and break point based debugging.It offers, breakpoints , checkpoints and multiple views which enhance the experience of debugging.\\

\textbf{Advantages}

\begin{enumerate}
  \item Can add any variable that one want to monitor to watch list.\
  \item Eclipse debugger allows to remote debug a process on any other machine .\
  \item One can move the current execution while executing . \
  \item One can step in and out of the code base based on whether it matters or not.\
  \item The debug perspective offers additional views that can be used to troubleshoot an application like Breakpoints, Variables, Debug, Console etc. \
  \item The Eclipse Debugging Platform helps developers debug by providing buttons in the toolbar and key binding shortcuts to control program execution. \
\end{enumerate}

\textbf{Disadvantages}
\begin{enumerate}
\item Debugging with eclipse will become difficult when the execution  of a particular function is time bound or if there is usage of sleep statements inside the program. \
\item It is difficult to monitor the programs that uses mutli threading . \
\end{enumerate}
\section*{Quality Attributes}
Quality attributes assessed while implementing the algorithm  are :\\
\begin{itemize}
  \item \textbf{Correctness:} Since I used taylor series for approximating the power function , I had to test with different number of iterations ranging from 10 to 100 to come to a conclusion on the optimum number of iterations to ensure the correctness of the function.Based on my tests , I came to a conclusion that to have minimum difference between expected output and actual output , the number of iterations needed is 13.
  \item \textbf{Efficiency:} As number of iterations increase, it increases the time of execution of the program but less no of iterations have a big impact on the correctness of the program.Hence to have a above average efficiency and a decent correctness I have chosen the no of iterations as 13.
  \item \textbf{Maintainability:} I have refactored the code and included the comments to improve the maintainability and understandability of the code.Have a test file to ensure that any changes made doesn't have an effect on the existing functionality of the program.
  \item \textbf{Robustness:} I have handled the situation where the user might give a wrong input such as a string in the place of a double .Hence the program doesn't crash and notifies the user with an appropriate error message.This increases the robustness of the program.
  \item \textbf{Usability:} I am packaging my program in an executable jar file so that other users can use my file without any difficulties.This increases the usability of the program.
\end{itemize}

\newpage


\end{document}