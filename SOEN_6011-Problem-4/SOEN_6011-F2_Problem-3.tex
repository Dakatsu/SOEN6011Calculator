
\documentclass[a4paper, 11pt]{report}
\usepackage[utf8]{inputenc}
\usepackage{titlesec}
\usepackage{fullpage} % changes the margin
\usepackage{graphicx} %package to manage images
\graphicspath{ {./images/} }
\usepackage{algorithm}
\usepackage{algpseudocode}

\begin{document}
\begin{titlepage}
\vspace*{0.7in}
\begin{center}
\begin{figure}[htb]
\begin{center}
\includegraphics[width=8cm]{univ_logo}
\end{center}
\end{figure}
\vspace*{0.3in}
\begin{Large}
\textbf{SOEN 6011 : SOFTWARE ENGINEERING PROCESSES} \\
\end{Large}
\vspace*{0.1in}
\begin{Large}
\textbf{SUMMER 2021} \\
\end{Large}
\vspace*{0.9in}
\begin{Large}
\textbf{SUPER CALCULATOR} \\
\end{Large}
\vspace*{0.9in}
\begin{Large} 


\textbf{PROBLEM - 3} \\
Pseudo-code and Algorithms\\
\end{Large}
\vspace*{0.9in}
\rule{80mm}{0.1mm}\\
\vspace*{0.1in}
\begin{large}
Authors \\
\vspace*{0.1in}
Rokeya Begum Keya\\
\vspace*{0.1in}
Kyle Taylor Lange\\
\vspace*{0.1in}
Sijie Min\\
\vspace*{0.1in}
Manimaran Palani\\ 
\vspace*{0.3in}
\date{\normalsize\today} 
\end{large}
\end{center}
\begin{center}
https://www.overleaf.com/project/610304de4e6b8d24f7c781b6\end{center}
\end{titlepage}
\tableofcontents
\newpage
\addcontentsline{toc}{section}{a) Decision on Pseudo-Code Format }
\newpage
\section*{Decision on Pseudo-Code Format}
\addcontentsline{toc}{section}{b \& c) Algorithm Description and Pseudo-Code }
Our Team conducted a brainstorming session and referred several resources \cite{standard} \cite{Janos} to decide on Pseudo code algorithm/pattern. As a conclusion, Everyone had agreed to make a pseudo code of their respective algoriths in the Algorithmicx (algpseudocode) \cite{algorithmicx} format available in Overleaf Latex.
\\\\\\
\textbf{Example of Algorithmicx (algpseudocode) Pseudo code Pattern}
\\
\begin{algorithm}
\caption{Algorithmicx (algpseudocode) Pseudo code Pattern} \label{alg:cap}
\begin{algorithmic}
\Require $n \geq 0$
\Ensure $y = x^n$
\State $y \gets 1$
\State $X \gets x$
\State $N \gets n$
\While{$N \neq 0$}
\If{$N$ is even}
    \State $X \gets X \times X$
    \State $N \gets \frac{N}{2}$  \Comment{This is a comment}
\ElsIf{$N$ is odd}
    \State $y \gets y \times X$
    \State $N \gets N - 1$
\EndIf
\EndWhile
\end{algorithmic}
\end{algorithm}

\pagebreak
\section*{Algorithm Description and Pseudo-Code}
\section*{\centering{PROBLEM 3 - F2: $tan(x)$}}
\normalsize {SOEN 6011 - Summer 2021} \hfill \textbf{Rokeya Begum Keya} \\
\textbf{ Software Engineering Processes}  \hfill \textbf{40183615} \\
\hfill Repository address : https://github.com/Dakatsu/SOEN6011Calculator
\\\\\\
\textbf{\\ \\ Technical Reasons for selecting Maclaurin Series: }
\begin{itemize}
\item There are many reasons for selecting Maclaurin Series for calculating the value of $tan(x)$ function. Below are some advantages for which I selected Maclaurin Series:
\end{itemize}
\textbf{Advantages: }
\begin{itemize}
\item Maclaurin series provides more approximate values for the tangent function.
\item The formula to calculate the value of $sin(x)$ and $cos(x)$ function to get the value of tangent function is easy to understand.\[tan(x) = \frac{sin(x)}{cos(x)}\]
\end{itemize}
\textbf{Disadvantages: }
\begin{itemize}
\item There is an another form of Maclaurin series to calculate the tangent function. For example: derivation of $tan(x)$ function. In this formula there are no use of $sin(x)$ and $cos(x)$ function. However, using this formula we can not get the approximate value of $tan(x)$ function.
\end{itemize}
\textbf{Therefore, I select the Maclaurin series of $sin(x)$ and $cos(x)$ to calculate the tangent function.}
\\
\textbf{\\ \\ Algorithm 1 - Maclaurin Series: } 
\begin{itemize}
\item In this project to calculate $tan(x)$ function, I select the Maclaurin Series.
Maclaurin series is just a special case of
taylor series where region near $x=0$.
\item The $tan(x)$ function's approximation is derived by the Maclaurin Series's explicit forms of $sin(x)$ and $cos(x)$.
\begin{equation} 
sin(x) = x-x^3/3!+x^5/5!-x^7/7!+.....
\end{equation}
\begin{equation} 
cos(x) = 1-x^2/2!+x^4/4!-x^6/6!+.....
\end{equation}
\item As, $tan(x)$ is an odd function, odd derivatives when x=0 of Maclaurin series are used to calculate $tan(x)$ function. \item The output are in integer and provide an approximate value for tangent function.
\end{itemize}

\textbf{Pseudo Code for Maclaurin Series}
\\\\\\\\\\\\\
\begin{algorithm}
\caption{Maclaurin Series} \label{alg:cap}
\begin{algorithmic}
\Require $retVal = 1$ AND $tmpResult = 1$ AND $i=1$
\Function{Explicit form}{cos(x)}
\For{$i \gets i+2$}
\State $value = (-1)*x*x/(i*(i+1))$\Comment{$Series for cos(x) $}
\State $tmpResult = tmpResult * value$
\If{check($value$)$<=EPS$}
    \EndIf   
    \State $retVal = retVal + tmpResult$ 
\EndFor \\
\Return{$retVal$}\Comment{$get value of cos(x)$}
\EndFunction
\Require $retVal = x$ AND $tmpResult = x$ AND $i=0$
\Function{Explicit form}{sin(x)}
\For{$i \gets i+1$}
\State $value = ((-1)*x*x/((2*i+2)*(2*i+3)))$\Comment{$Series for sin(x) $}
\State $tmpResult = tmpResult * value$
\If{check($value$)$<=EPS$}
    \EndIf   
    \State $retVal = retVal + tmpResult$ 
\EndFor \\
\Return{$retVal$}\Comment{$get value of sin(x)$}
\EndFunction
\Require $x = Rad(x)$ AND $SinVal = retVal$ AND $CosVal= reVal$
\Function{Calculate}{tan(x)}
\If{$SinVal< EPSvalMini$}
\\\ \Return 0
    \EndIf   
\If{$CosVal< EPSvalMini$}
\\\    \Return $undefined$
    \EndIf \\      
\Return{$SinVal$/$CosVal$}\Comment{$calculation for tan(x)$}
\EndFunction
\State $result \get $ \Call{$tan(x)$}{}
\end{algorithmic}
\end{algorithm}
\pagebreak

\section*{\centering{PROBLEM 3 - F3: Hyperbolic Sine, $sinh(x)$}}
\normalsize {SOEN 6011 - Summer 2021} \hfill \textbf{Kyle Taylor Lange} \\
\textbf{ Software Engineering Processes}  \hfill \textbf{27627696} \\https://www.overleaf.com/project/610304de4e6b8d24f7c781b6
\hfill Repository address : https://github.com/Dakatsu/SOEN6011Calculator
\\\\\\

\pagebreak

\section*{\centering{PROBLEM 3 - F5}}
\normalsize {SOEN 6011 - Summer 2021} \hfill \textbf{Sijie Min} \\
\textbf{ Software Engineering Processes}  \hfill \textbf{401*****} \\
\hfill Repository address : https://github.com/Dakatsu/SOEN6011Calculator
\\\\\\\\\\
 \begin{center} Team please add your content here \end{center}
\pagebreak

\section*{\centering{PROBLEM 3 - F7 : \(x^y\)}}
\normalsize {SOEN 6011 - Summer 2021} \hfill \textbf{Manimaran Palani} \\
\textbf{ Software Engineering Processes}  \hfill \textbf{40167543} \\
\hfill Repository address : https://github.com/Dakatsu/SOEN6011Calculator
\\
\textbf{\\ \\ Algorithm 2: Montgomery's Ladder Technique} 
\begin{itemize}
\item Montgomerym's ladder technique addresses defence against side-channel attacks for exponentiation computation. \\\\
The algorithm prevents the recovery of the exponent involved in the computation which could possibly benefit an attacker
\item The algorithm performs a fixed sequence of operations (up to log n): a multiplication and squaring takes place for each bit in the exponent, regardless of the bit's specific value.
\end{itemize}
\vspace*{0.2in}
\setlength{\tabcolsep}{18pt}
\renewcommand{\arraystretch}{1.5}
\begin{tabular}{ |p{6cm}|p{6cm}| }
\hline
\textbf{Advantages} & \textbf{Disadvantages}
\\ \hline 
It addresses the concern of MIM(Middle Man attack) observing the sequence of squaring and multiplications can (partially) recover the exponent involved in the computation. & Cache timing attacks are not yet protected and memory access latency might still be observable to an attacker\\
\hline
\end{tabular} 

\textbf{\\ \\ Algorithm 3: Taylor series}\\ \\Taylor series is a representation of a function as an infinite sum of terms that are calculated from the values of the function's derivatives at a single point.
\begin{equation} \label{evalpow}
x^y= e^{y\ln x}
\end{equation}
\ref{evalpow} evaluation of $x^y$. Here, e is a mathematical constant approximately equal to  2.71828
\begin{equation} \label{extaylor}
e^x = 1 + x/1! + x^2/2! + x^3/3! + ...... 
\end{equation}
\ref{extaylor} express $e^x$ using Taylor Series
\begin{equation} \label{altextaylore}
e^x = 1 + (x/1) (1 + (x/2) (1 + (x/3) (........) ) ) 
\end{equation}
\ref{altextaylore} The series \ref{extaylor} can be re-written as above
\begin{equation} \label{logtaylor}
log(1+x) = x-x^2/2 + x^3/3- ... 
\end{equation}
\ref{logtaylor} express ln x using Taylor Series\\ \\

\setlength{\tabcolsep}{18pt}
\renewcommand{\arraystretch}{1.5}
\begin{tabular}{ |p{6cm}|p{6cm}| }
\hline
\textbf{Advantages} & \textbf{Disadvantages}\\ \hline 
Very useful for derivations
 & Successive terms get very complex and hard to derive\\
\hline
Can be used to get theoretical error bounds &Truncation error tends to grow rapidly away from expansion point\\
\hline
Object Reference Model parameters embedded as variables & fsdfds\\
\hline
Power series can be inverted to yield the inverse function & Almost always not as efficient as curve fitting or direct approximation\\
\hline
\end{tabular} \\ \\ 
\begin{algorithm}
\caption{Montgomery's ladder Exponential Function}\label{exp3}
\begin{algorithmic}[1]
\Require $x\textsubscript{1} = $x; $x\textsubscript{2} = $x^2$
\State For $i=k-2$ to 0 do$
\If{n \textsubscript{i}=0} 
\State  x2 = x1 * x2; x1 = x12
\State else
\State   x1 = x1 * x2; x2 = x22
\Return x1
\end{algorithmic}
\end{algorithm}

\begin{algorithm}
\caption{Exponentiation by Taylor Series}\label{exp1}
\begin{algorithmic}[1]
\Require $x \neq 0$ AND $y > 0$

\Function{logarithm}{$n$}\Comment{$algorithm for log(n)$}
\State $sum\gets 0$
\While{$n > 1$}
    \State $n\gets n/e$\Comment{e is a constant approximately equal to 2.71828}
    \State $y \gets y+1 $
\EndWhile \\
\Return $y$
\EndFunction

\Function{exponential}{$x$}\Comment{$algorithm for e^x$}
\State $sum\gets 1$
\State $n\gets 10$
\For{$i\gets n-1$, 1}
\State $sum\gets 1+ x * sum / i$
\EndFor \\
\Return $sum$
\EndFunction

\State $logx \gets $\Call{logarithm}{x}
\State $result \gets $\Call{exponential}{y*logx}
\end{algorithmic}
\end{algorithm}

\begin{thebibliography}{}
\bibitem{algorithmicx} 
Algorithmicx (algpseudocode)
\\\texttt{https://www.overleaf.com/latex/examples/pseudocode-example/pbssqzhvktkj}
\bibitem{standard} 
Pseudo-Code Standard
\\\texttt{http://users.csc.calpoly.edu/\~jdalbey/SWE/pdl\_std.html}
\bibitem{Janos} 
Szasz Janos, The algorithmicx package
\\\texttt{http://tug.ctan.org/macros/latex/contrib/algorithmicx/algorithmicx.pdf}
\bibitem{Peter} 
 Montgomery, Peter L. (1987). "Speeding the Pollard and Elliptic Curve Methods of Factorization" (PDF)
\\\texttt{https://www.ams.org/journals/mcom/1987-48-177/S0025-5718-1987-0866113-7/S0025-5718-1987-0866113-7.pdf}
\end{thebibliography}
\end{document}