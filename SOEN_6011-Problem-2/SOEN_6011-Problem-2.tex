
\documentclass[a4paper, 11pt]{report}
\usepackage[utf8]{inputenc}
\usepackage{titlesec}
\usepackage{fullpage} % changes the margin
\usepackage{graphicx} %package to manage images
\graphicspath{ {./images/} }

\begin{document}
\begin{titlepage}
\vspace*{0.7in}
\begin{center}
\begin{figure}[htb]
\begin{center}
\includegraphics[width=8cm]{univ_logo}
\end{center}
\end{figure}
\vspace*{0.3in}
\begin{Large}
\textbf{SOEN 6011 : SOFTWARE ENGINEERING PROCESSES} \\
\end{Large}
\vspace*{0.1in}
\begin{Large}
\textbf{SUMMER 2021} \\
\end{Large}
\vspace*{0.9in}
\begin{Large}
\textbf{SUPER CALCULATOR} \\
\end{Large}
\vspace*{0.9in}
\begin{Large} 


\textbf{PROBLEM - 2} \\
Requirements\\\footnotesize{ISO/IEC/IEEE} 29148 Standard \\
\end{Large}
\vspace*{0.9in}
\rule{80mm}{0.1mm}\\
\vspace*{0.1in}
\begin{large}
Authors \\
\vspace*{0.1in}
Rokeya Begum Keya\\
\vspace*{0.1in}
Kyle Taylor Lange\\
\vspace*{0.1in}
Sijie Min\\
\vspace*{0.1in}
Manimaran Palani\\ 
\vspace*{0.3in}
\date{\normalsize\today} 
\end{large}
\end{center}
\begin{center}
https://www.overleaf.com/project/610304de4e6b8d24f7c781b6\end{center}
\end{titlepage}

\newpage
\section*{\centering{PROBLEM 2 - F2: $tan(x)$}}
\normalsize {SOEN 6011 - Summer 2021} \hfill \textbf{Rokeya Begum Keya} \\
\textbf{ Software Engineering Processes}  \hfill \textbf{40183615} \\
\hfill Repository address : https://github.com/Dakatsu/SOEN6011Calculator
\\\\\\
\section*{Assumption:} 
The value of $tan(x)$ function is real number. Moreover, the calculation of $tan(x)$ function is done in radians.
\\\
\section*{Requirements:}\cite{ReqView}\cite{29148}\\
The current section describes the requirements to implement the function $tan(x)$.
\\\\\\
\textbf{Requirement Id : F2-R1}\\\\
\begin{tabular}{ll}
\textbf{Overview} & Input $x$ into $tan(x)$ function. \\
\textbf{Version} & 1.0 \\
\textbf{Description} & 
\begin{tabular}[c]{@{}l@{}} User should give integer (degree) value as input. The program will give 
\\the approximate integer value of $tan(x)$.
\end{tabular} \\
\textbf{Owner} & Rokeya Begum Keya \\
\textbf{Priority} & High \\
\textbf{Type} & Functional \\
\textbf{Difficulty} & Medium \\
\textbf{Verification Method} &                  \end{tabular}
\\\\\\\\\\
\textbf{Requirement Id : F2-R2}\\\\
\begin{tabular}{ll}
\textbf{Overview} & Input $x$ into $tan(x)$ function. \\
\textbf{Version} & 1.0 \\
\textbf{Description} & 
\begin{tabular}[c]{@{}l@{}} If User gives any value out of domain. The output will show error.
\end{tabular} \\
\textbf{Owner} & Rokeya Begum Keya \\
\textbf{Priority} & High \\
\textbf{Type} & Functional \\
\textbf{Difficulty} & Medium \\
\textbf{Verification Method} &                                          \end{tabular}
\\\\\\\\\\\\\\\\\\\\\
\textbf{Requirement Id : F2-R3}\\\\
\begin{tabular}{ll}
\textbf{Overview} & Input $x$ into $tan(x)$ function. \\
\textbf{Version} & 1.0 \\
\textbf{Description} & 
\begin{tabular}[c]{@{}l@{}} If User gives any integer value which is out of range. 
\\The output will be undefined and will show error.
\end{tabular} \\
\textbf{Owner} & Rokeya Begum Keya \\
\textbf{Priority} & High \\
\textbf{Type} & Functional \\
\textbf{Difficulty} & Medium \\
\textbf{Verification Method} &                  \end{tabular}
\\\\\\\\\\
\textbf{Requirement Id : F2-R4}\\\\
\begin{tabular}{ll}
\textbf{Overview} & Input $x$ into $tan(x)$ function. \\
\textbf{Version} & 1.0 \\
\textbf{Description} & 
\begin{tabular}[c]{@{}l@{}} For the input, for which $cos(x) = 0$, then, the output
\\will be undefined and will show "undefined".
\end{tabular} \\
\textbf{Owner} & Rokeya Begum Keya \\
\textbf{Priority} & High \\
\textbf{Type} & Functional \\
\textbf{Difficulty} & Medium \\
\textbf{Verification Method} &                  \end{tabular}
\\\\\\\\\\
\textbf{Requirement Id : F2-R5}\\\\
\begin{tabular}{ll}
\textbf{Overview} & Input $x$ into $tan(x)$ function. \\
\textbf{Version} & 1.0 \\
\textbf{Description} & 
\begin{tabular}[c]{@{}l@{}} If the user gives an input of $tan(90^\circ)$, 
\\then, the output will be "undefined".
\end{tabular} \\
\textbf{Owner} & Rokeya Begum Keya \\
\textbf{Priority} & High \\
\textbf{Type} & Functional \\
\textbf{Difficulty} & Medium \\
\textbf{Verification Method} &                                        \end{tabular}
\\\\\\\\\\\\\\\\\\\\\\\\
\textbf{Requirement Id : F2-R6}\\\\
\begin{tabular}{ll}
\textbf{Overview} & Availability \\
\textbf{Version} & 1.0 \\
\textbf{Description} & 
\begin{tabular}[c]{@{}l@{}} The system may provide the calculation to the user within four seconds.
\end{tabular} \\
\textbf{Owner} & Rokeya Begum Keya \\
\textbf{Priority} & High \\
\textbf{Type} & Functional \\
\textbf{Difficulty} & Medium \\
\textbf{Verification Method} &                  \end{tabular}
\pagebreak

\section*{\centering{PROBLEM 2 - F3: Hyperbolic Sine, $sinh(x)$}}
\normalsize {SOEN 6011 - Summer 2021} \hfill \textbf{Kyle Taylor Lange} \\
\textbf{ Software Engineering Processes}  \hfill \textbf{27627696} \\https://www.overleaf.com/project/610304de4e6b8d24f7c781b6
\hfill Repository address : https://github.com/Dakatsu/SOEN6011Calculator
\\\\\\

\pagebreak

\section*{\centering{PROBLEM 2 - F5}}
\normalsize {SOEN 6011 - Summer 2021} \hfill \textbf{Sijie Min} \\
\textbf{ Software Engineering Processes}  \hfill \textbf{401*****} \\
\hfill Repository address : https://github.com/Dakatsu/SOEN6011Calculator
\\\\\\\\\\
 \begin{center} Team please add your content here \end{center}
\pagebreak

\section*{\centering{PROBLEM 2 - F7 : \(x^y\)}}
\normalsize {SOEN 6011 - Summer 2021} \hfill \textbf{Manimaran Palani} \\
\textbf{ Software Engineering Processes}  \hfill \textbf{40167543} \\
\hfill Repository address : https://github.com/Dakatsu/SOEN6011Calculator
\\
\section*{Requirements and Assumptions}\cite{ReqView}\cite{29148}\\
The current section describes the requirements and assumptions to implement the function \(x^y\).
\\\\\\
\textbf{Explicit Assumption :}  The transcendental function \(x^y\) will be accurate and accepts input which comprises of rational and irrational numbers.
\\\\\\\\
\textbf{Requirement Id : F7-R1}\\\\
\begin{tabular}{ll}
\textbf{Overview} & X(0) to the power of Y(0) \\
\textbf{Version} & 1.0 \\
\textbf{Description} & 
\begin{tabular}[c]{@{}l@{}}If the user gives an input for X as Zero and input for Y as Zero.\\The function may return the 1 as output.
\end{tabular} \\
\textbf{Owner} & Manimaran Palani \\
\textbf{Priority} & High \\
\textbf{Type} & Functional \\
\textbf{Difficulty} & Medium \\
\textbf{Verification} Method \\
\end{tabular}
\\\\\\\\\\
\textbf{Requirement Id : F7-R2}\\\\
\begin{tabular}{ll}
\textbf{Overview} & X(0) to the power of Y (Positive Numbers) \\
\textbf{Version} & 1.0 \\
\textbf{Description} & 
\begin{tabular}[c]{@{}l@{}}If the user gives an input for X as zero and input for Y as
\\any positive Number. The function may return zero as output.
\end{tabular} \\
\textbf{Owner} & Manimaran Palani \\
\textbf{Priority} & High \\
\textbf{Type} & Functional \\
\textbf{Difficulty} & Medium \\
\textbf{Verification Method} &                                    \end{tabular}
\\\\\\\\\\\\\\\\\\
\textbf{Requirement Id : F7-R3}\\\\
\begin{tabular}{ll}
\textbf{Overview} & X(0) to the power of Y (Negative Numbers) \\
\textbf{Version} & 1.0 \\
\textbf{Description} & 
\begin{tabular}[c]{@{}l@{}}If the user gives an input for X as zero and input for Y as any
\\ Negative Number.The function may return infinity as output.
\end{tabular} \\
\textbf{Owner} & Manimaran Palani \\
\textbf{Priority} & High \\
\textbf{Type} & Functional \\
\textbf{Difficulty} & Medium \\
\textbf{Verification Method} &                                    \end{tabular}
\\\\\\\\\\\\\\\\
\textbf{Requirement Id : F7-R4}\\\\
\begin{tabular}{ll}
\textbf{Overview} & X(Positive Number) to the power of Y (0) \\
\textbf{Version} & 1.0 \\
\textbf{Description} & 
\begin{tabular}[c]{@{}l@{}}If the user gives an input for X of any positive number and\\input for Y as Zero.The function may return 1 as the output.
\end{tabular} \\
\textbf{Owner} & Manimaran Palani \\
\textbf{Priority} & High \\
\textbf{Type} & Functional \\
\textbf{Difficulty} & Medium \\
\textbf{Verification Method} &                                                          \end{tabular}
\\\\\\\\\\\\\\\\
\textbf{Requirement Id : F7-R5}\\\\
\begin{tabular}{ll}
\textbf{Overview} & X(Negative Number) to the power of Y (0) \\
\textbf{Version} & 1.0 \\
\textbf{Description} & 
\begin{tabular}[c]{@{}l@{}}If the user gives an input for X of any negative number and
\\input for Y as Zero.The function may return -1 as the output.
\end{tabular} \\
\textbf{Owner} & Manimaran Palani \\
\textbf{Priority} & High \\
\textbf{Type} & Functional \\
\textbf{Difficulty} & Medium \\
\textbf{Verification Method} &  
\end{tabular}
\\\\\\\\\\\\\\\\
\textbf{Requirement Id : F7-R6}\\\\
\begin{tabular}{ll}
\textbf{Overview} & X(Negative Number) to the power of Y (Positive or Negative Number) \\
\textbf{Version} & 1.0 \\
\textbf{Description} & 
\begin{tabular}[c]{@{}l@{}}If the user gives an input for X as any negative number and input \\for Y as positive or negative number. The function may return
\\negative number as the output.
\end{tabular} \\
\textbf{Owner} & Manimaran Palani \\
\textbf{Priority} & High \\
\textbf{Type} & Functional \\
\textbf{Difficulty} & Medium \\
\textbf{Verification Method} &
\end{tabular}
\\\\\\\\\\\\\\
\textbf{Requirement Id : F7-R7}\\\\
\begin{tabular}{ll}
\textbf{Overview} & X(Positive Number) to the power of Y  (Positive or Negative Number) \\
\textbf{Version} & 1.0 \\
\textbf{Description} & \begin{tabular}[c]{@{}l@{}}If the user gives an input for X as any positive number and input \\for Y as positive or negative numbers. The function may return\\positive number as the output.\end{tabular} \\
\textbf{Owner} & Manimaran Palani \\
\textbf{Priority} & High \\
\textbf{Type} & Functional \\
\textbf{Difficulty} & Medium \\
\textbf{Verification Method} &                                     \end{tabular}
\\\\\\\\\\\\
\textbf{Requirement Id : F7-R8}\\\\
\begin{tabular}{ll}
\textbf{Overview} & Availability \\
\textbf{Version} & 1.0 \\
\textbf{Description} & 
\begin{tabular}[c]{@{}l@{}}The system may provide the response with output to the user \\ within finite time. \\
\end{tabular} \\
\textbf{Owner} & Manimaran Palani \\
\textbf{Priority} & High \\
\textbf{Type} & Non-Functional \\
\textbf{Difficulty} & Medium \\
\textbf{Verification Method} &
\end{tabular}
\begin{thebibliography}{}
\bibitem{ReqView} 
ReqView : Nykamp DQ: Requirements Specification Templates
\\\texttt{https://www.reqview.com/doc/iso-iec-ieee-29148-templates}
\bibitem{29148} 
29148-2018-ISO/IEC/IEEE International Standard-Systems and software engineering-Life cycle processes-Requirements engineering,
\\\texttt{https://standards.ieee.org/standard/29148-2018.html}
\end{thebibliography}
\end{document}